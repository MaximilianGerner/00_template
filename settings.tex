%%%%%%%%%%%%%%%%%%%%%%%%%%%%%%%%%%%%%%%%%%%%%%%%%%%%%%%%%%%%%%%%%%%%%%%%%%%%%%%%
%%%
%%% define the titlepage
%%%
%%%
%%% Notes: substitute variables are used because \maketitle clears 
%%% \title, \author etc.
%%%

%%% subject
\def\getsubject{Bericht 404}   
\subject{\getsubject}   %% subject which appears above titlehead
% \titlehead{} %% special heading for the titlepage

%%% title
\def\gettitle{Untersuchung einer Antriebswelle 404}
\title{\gettitle}

%%% subtitle
\def\getsubtitle{Das ist ein Subtitle 404}
\subtitle{\getsubtitle\\[60pt]}

%%% author(s) use \and for more than one author (search for workaround with the warnings thrown by hyperrefsetup
\def\getauthor{Maximilian Gerner}
\author{\getauthor}

%%% date
\def\getdate{Graz, Wintersemester 2023}
\date{\getdate}

% \publishers{}

% \thanks{} %% use it instead of footnotes (only on titlepage)

% \dedication{} %% generates a dedication-page after titlepage


%%%%%%%%%%%%%%%%%%%%%%%%%%%%%%%%%%%%%%%%%%%%%%%%%%%%%%%%%%%%%%%%%%%%%%%%%%%%%%%%
%%%
%%% packages
%%%

%%%
%%% encoding and language set
%%%

%%% ngerman: language set to new-german
\usepackage{ngerman}

%%% babel: language set (can cause some conflicts with package ngerman)
%%%        use it only for multi-language documents or non-german ones
%\usepackage[ngerman]{babel}

%%% inputenc: coding of german special characters
\usepackage[utf8]{inputenc}

%%% for Linux:
% \usepackage[latin1]{inputenc}

%%% fontenc, ae, aecompl: coding of characters in PDF documents
\usepackage[T1]{fontenc}
\usepackage{ae, aecompl}

%%%
%%% technical packages
%%%

%%% amsmath, amssymb, amstext: support for mathematics
\usepackage{amsmath, amssymb, amstext}

%%% mathtools: supports more mathmatical symbols
\usepackage{mathtools}

%%% psfrag: replace PostScript fonts
%\usepackage{psfrag}

%%% listings: include programming code
%\usepackage{listings}

%%% units: technical units
\usepackage{units}

%%% enumitem and pifont: add more symbols to use in bullet lists
\usepackage{enumitem}
\usepackage{pifont}

%%% float: manage where to put included images and add priorities
\usepackage{float}


%%%
%%% layout
%%%

%%% scrlayer-scrpage: KOMA heading and footer
%%% Note: if you don't use this package, please remove 
%%%       \pagestyle{scrheadings} and corresponding settings
%%%       below too.
\usepackage{scrlayer-scrpage}






%%%
%%% PDF
%%%

\newif\ifpdf
  \ifx\pdfoutput\undefined
     \pdffalse
  \else
     \pdfoutput=1
     \pdftrue
  \fi

%%% Should be LAST usepackage-call!
%%% For docu on that, see reference on package ``hyperref''
\Ifpdfoutput{%   (definitions for using pdflatex instead of latex)

  %%% graphicx: support for graphics
  \usepackage[pdftex]{graphicx}

  \pdfcompresslevel=9

  %%% hyperref (hyperlinks in PDF): for more options or more detailed
  %%%          explanations, see the documentation of the hyperref-package
  \usepackage[%
    %%% general options
    %pdftex=true,      %% sets up hyperref for use with the pdftex program
    %plainpages=false, %% set it to false, if pdflatex complains: ``destination with same identifier already exists''
    %
    %%% extension options
    backref=page,      %% if page, adds a backlink text to the end of each item in the bibliography that points to the page where the source is cited
    pagebackref=false, %% if true, creates backward references as a list of page numbers in the bibliography
    colorlinks=true,   %% turn on colored links (true is better for on-screen reading, false is better for printout versions)
    %
    %%% PDF-specific display options
    bookmarks=true,          %% if true, generate PDF bookmarks (requires two passes of pdflatex)
    bookmarksopen=false,     %% if true, show all PDF bookmarks expanded
    bookmarksnumbered=false, %% if true, add the section numbers to the bookmarks
    %pdfstartpage={1},        %% determines, on which page the PDF file is opened
    pdfpagemode=UseNone         %% None, UseOutlines (=show bookmarks), UseThumbs (show thumbnails), FullScreen
  ]{hyperref}


  %%% provide all graphics (also) in this format, so you don't have
  %%% to add the file extensions to the \includegraphics-command
  %%% and/or you don't have to distinguish between generating
  %%% dvi/ps (through latex) and pdf (through pdflatex)
  \DeclareGraphicsExtensions{.pdf}

}{%else   (definitions for using latex instead of pdflatex)

  \usepackage[dvips]{graphicx}

  \DeclareGraphicsExtensions{.eps}

  \usepackage[%
    dvips,           %% sets up hyperref for use with the dvips driver
    colorlinks=false %% better for printout version; almost every hyperref-extension is eliminated by using dvips
  ]{hyperref}

}


%%% sets the PDF-Informations options
%%% (see fields in Acrobat Reader: ``File -> Document properties -> Summary'')
%%% Note: this method is better than as options of the hyperref-package (options are expanded correctly)
\hypersetup{
  pdftitle={\gettitle}, %%
  pdfauthor={\getauthor}, %%
  pdfsubject={\getsubject}, %%
  pdfcreator={Accomplished with LaTeX2e and pdfLaTeX with hyperref-package.}, %% 
  pdfproducer={}, %%
  pdfkeywords={} %%
}


%%%%%%%%%%%%%%%%%%%%%%%%%%%%%%%%%%%%%%%%%%%%%%%%%%%%%%%%%%%%%%%%%%%%%%%%%%%%%%%%
%%%
%%% user defined commands
%%%

%%% \mygraphics{}{}{}
%% usage:   \mygraphics{width}{filename_without_extension}{caption}
%% example: \mygraphics{0.7\textwidth}{rolling_grandma}{This is my grandmother on inlinescates}
%% requires: package graphicx
%% provides: including centered pictures/graphics with a boldfaced caption below
%% 
\newcommand{\mygraphics}[3]{
  \begin{center}
    \includegraphics[width=#1, keepaspectratio=true]{#2} \\
    \textbf{#3}
  \end{center}
}


%% \vecttwo{}
%% 
%% usage:       \vecttwo{a, b}
%% requires:    amsmath, accents
\newcommand{\vecttwo}[2]{\begin{pmatrix} #1 \\ #2  \end{pmatrix}}


%% \*
%% shortcut for \cdot
%% usage \*
\renewcommand{\*}{\cdot}


%%%%%%%%%%%%%%%%%%%%%%%%%%%%%%%%%%%%%%%%%%%%%%%%%%%%%%%%%%%%%%%%%%%%%%%%%%%%%%%%
%%%
%%% set heading and footer
%%%

%%% scrheadings default: 
%%%      footer - middle: page number
\pagestyle{scrheadings}

%%% user specific
%%% usage:
%%% \position[heading/footer for the titlepage]{heading/footer for the rest of the document}

%%% heading - left
% \ihead[]{}

%%% heading - center
% \chead[]{}

%%% heading - right
 \ohead[]{\gettitle}

%%% footer - left
% \ifoot[]{}

%%% footer - center
% \cfoot[]{}

%%% footer - right
% \ofoot[]{}
